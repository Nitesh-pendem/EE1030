%iffalse
\let\negmedspace\undefined
\let\negthickspace\undefined
\documentclass[journal,12pt,onecolumn]{IEEEtran}
\usepackage{cite}
\usepackage{amsmath,amssymb,amsfonts,amsthm}
\usepackage{algorithmic}
\usepackage{multicol}
\usepackage{circuitikz}
\usepackage{tikz}
\usepackage{graphicx}
\usepackage{textcomp}
\usepackage{xcolor}
\usepackage{txfonts}
\usepackage{listings}
\usepackage{enumitem}
\usepackage{mathtools}
\usepackage{gensymb}
\usepackage{comment}
\usepackage[breaklinks=true]{hyperref}
\usepackage{tkz-euclide} 
\usepackage{listings}
\usepackage{gvv}

%\def\inputGnumericTable{}                                
\usepackage[latin1]{inputenc}                                
\usepackage{color}                                            
\usepackage{array}                                            
\usepackage{longtable}                                       
\usepackage{calc}                                             
\usepackage{multirow}                                         
\usepackage{hhline}                                           
\usepackage{ifthen}                                           
\usepackage{lscape}
\usepackage{tabularx}
\usepackage{array}
\usepackage{float}
\newtheorem{theorem}{Theorem}[section]
\newtheorem{problem}{Problem}
\newtheorem{proposition}{Proposition}[section]
\newtheorem{lemma}{Lemma}[section]
\newtheorem{corollary}[theorem]{Corollary}
\newtheorem{example}{Example}[section]
\newtheorem{definition}[problem]{Definition}
\newcommand{\BEQA}{\begin{eqnarray}}
\newcommand{\EEQA}{\end{eqnarray}}
\newcommand{\define}{\stackrel{\triangle}{=}}
\theoremstyle{remark}


% Marks the beginning of the document
\begin{document}
\bibliographystyle{IEEEtran}
\vspace{3cm}

\title{\textbf{2013-AE-1-13}}
\author{AI24BTEC11026 - pendem nitesh sri satya}
\maketitle
\bigskip

\renewcommand{\thefigure}{\theenumi}
\renewcommand{\thetable}{\theenumi}
\setlength{\columnsep}{2.5em}
\begin{enumerate}
[start=1]
\item The directional derivative of the function $f(x,y)=\frac{x^2+xy^2}{\sqrt{5}}$ in the direction $\overrightarrow{a}=2 \hat{i}-4 \hat{j}$ at $(x,y)=(1,1)$ is 
\begin{enumerate}
    \item $-\frac{1}{\sqrt{5}}$
    \item $-\frac{2}{\sqrt{5}}$
    \item $0$
    \item $-\frac{1}{5} $\\
\end{enumerate}

\item The value of $\int_4^5 \frac{x+2}{x^2+4x-21} dx$ is 
\begin{enumerate}
    \item $ln\sqrt{\frac{24}{11}}$
    \item $ln\sqrt{\frac{12}{11}}$
    \item $ln\sqrt{2}$
    \item $ln(\frac{12}{11})$\\
\end{enumerate}

\item At $x=0$ the function $y=\abs{x}$ is
\begin{enumerate}
    \item continuous but not differentiable
    \item continuous and differentiable
    \item not continuous but differentiable
    \item not continuous and not differentiable \\
\end{enumerate}

\item one of the eigenvectors of the matrix \\ $A=\myvec{1 & -1 & 0\\ 0 & 1 & -1\\-1 & 0 & 1 }$ is $v=  \myvec{1 \\ 1 \\ 1}$ \\ The corresponding eigenvalue is \underline{\hspace{1cm}}.\\

\item which one of the following is the most stable configuration of an airplane in roll?
\begin{enumerate}
    \item Sweep back, anhedral and low ring
    \item Sweep forward, dihedral and low wing
    \item Sweep forward, anhedral and high wing
    \item Sweep back, dihedral and high wing \\
\end{enumerate}

\item which one of the following flight instruments is used on an aircraft to determine its attitude in flight?
\begin{enumerate}
    \item Vertical speed indicator
    \item Altimeter
    \item Artificial Horizon
    \item Turn-bank indicator \\
\end{enumerate}

\item A supersonic airplane is expected to fly at both subsonic and supersonic speeds during its whole flight course. which one of the following statements is TRUE?
\begin{enumerate}
    \item Airplane will experience less stability in pitch at supersonic speeds than at subsonic speeds
    \item Airplane will feel no change in pitch stability
    \item Airplane will experience more stability in pitch at supersonic speeds than at subsonic speeds
    \item pitch stability cannot be inferred from the information given \\
\end{enumerate}

\item which one of the following is favorable for an airplane operation?
\begin{enumerate}
    \item Tail wind in cruise and head wind in landing
    \item Tail wind both in cruise and landing
    \item head wind both in cruise and landing
    \item Head wind in cruise and tail wind in landing \\
\end{enumerate}

\item which one of the following is TRUE with respect to phugoid mode of an aircraft?
\begin{enumerate}
    \item Frequency is directly proportional to flight speed
    \item Frequency is inversely proportional to flight speed
    \item Frequency is directly proportional to the square root of flight speed
    \item Frequency is inversely proportional to the square root of flight speed \\
\end{enumerate}

\item the $x$ and $y$ velocity components of a two dimensional flow fled are, $u=\frac{cy}{x^2+y^2},v=\frac{cx}{x^2+y^2}$ where $c$ is a constant. The streamlines are a family of 
\begin{enumerate}
    \item hyperbolas
    \item parabolas
    \item ellipses
    \item circles \\
\end{enumerate}

\item which one of the following statements is NOT TRUE for a supersonic flow?
\begin{enumerate}
   \item over a gradual expansion, entropy remains constant
    \item over a gradual expansion corner, entropy can increase
     \item over a gradual compression, entropy can remain constant
     \item over a sharp compression corner, entropy increases \\
\end{enumerate}

\item consider a compressible flow where an elemental volume of the fluid is $\delta \rho$, moving with velocity $\overrightarrow{V}$. which one of the following expressions is TRUE?
\begin{enumerate}
    \item $\nabla \cdot \overrightarrow{V}=\frac{1}{\delta \rho} \frac{D\delta \rho}{Dt}$
    \item $\nabla \cdot (\nabla \times \overrightarrow{V})=\frac{1}{\delta \rho} \frac{D\delta \rho}{Dt}$
    \item $\nabla \cdot \frac{D\overrightarrow{V}}{Dt}=\frac{1}{\delta \rho} \frac{D\delta \rho}{Dt}$
    \item $\overrightarrow{V} \cdot (\nabla \times \overrightarrow{V})=\frac{1}{\delta \rho} \frac{D\delta \rho}{Dt}$ \\
\end{enumerate}

\item consider a thin flat plate airfoil at a small angle $\alpha$ to an oncoming supersonic stream of air. Assuming the flow to be inviscid, $\frac{C_d}{c_1^2}$ is
\begin{enumerate}
    \item zero
    \item independent of $\alpha$
    \item proportional to $\alpha$
    \item proportional to $\alpha^2$ \\
\end{enumerate}

\end{enumerate}
\end{document}

